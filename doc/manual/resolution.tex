\chapter{Numerical resolution parameters}
\label{ch:resolution}

Results from \sfincs~should only be believed if you are confident they
are converged with respect to the numerical resolution parameters
\Ntheta, \Nzeta, \Nxi, and \Nx.  That is, you want to be sure the
physics output of the code does not change significantly when any of
these parameters are increased. The values of \Ntheta, \Nzeta, \Nxi,
and \Nx~required for convergence depend strongly on the magnetic
geometry and collisionality, with modest dependence also on the radial
electric field. It is strongly recommended that you test for
convergence with respect to \Ntheta, \Nzeta, \Nxi, and \Nx~whenever
beginning \sfincs~calculations for a new scenario.

Note that ``convergence'' in this sense (convergence with respect to resolution parameters assuming the discretized
system is solved exactly) is separate from the convergence of GMRES/KSP.

\section{Relatively unimportant resolution parameters}

There are several resolution parameters which are almost never the limiting factor
for convergence, and so which almost never need to be adjusted.
These parameters and good values for them are {\ttfamily NL=4}, 
{\ttfamily solverTolerance} = $10^{-6}$,
{\ttfamily xMax=5.0},
and {\ttfamily NxPotentialsPerVth=40.0}.  
(These values are set as the defaults.)
The latter two of these parameters are in fact ignored
for the recommended and default {\ttfamily xGridMode} setting, 5.

\section{General suggestions}

The time and memory requirements of the code increase significantly
when \Ntheta, \Nzeta, \Nxi, or \Nx~are increased. Therefore, you
probably want to only scan one of these four parameters at a time
(rather than increasing two or more of them simultaneously) when
testing for convergence. (This recommended approach is the one taken
in \sfincsScan~automated convergence scans, discussed in section
\ref{sec:convergence})

When the mean-free-path is shorter than the parallel length scale of the equilibrium,
({\ttfamily nuPrime} $\ge$ 1), the parameters required for convergence
do not depend much on collisionality. In the opposite
limit in which the mean-free-path is longer than the parallel length scale of the equilibrium,
({\ttfamily nuPrime} $\le$ 1)
values of
\Nzeta~and \Nxi~required for convergence increase dramatically
as collisionality decreases. The required value of \Ntheta~increases as well, but often
not quite as dramatically. The \Nx~required for convergence does not depend
much on collisionality, though typically the required value increases
slightly with collisionality at high collisionality

The \Nx~required for convergence may need to increase slightly with the number of species.
Typically you can expect to use \Nx=5-8.

The resolution parameters do not need to vary much with the radial
electric field as long as the electric field is below about 1/3 of the resonant value.
(In the notation of \cite{sfincsPaper}, when $E_*<1/3$).
For almost all experimentally relevant situations (except for HSX where $T_i/T_e$ is extremely small),
the electric field is far below the resonance, in which case you should not need to vary the resolution
parameters with the electric field.
However, if you do approach the $E_r$ resonance, \Nx~will likely need to be increased.

\section{Convergence testing}
\label{sec:convergence}

\section{Typical resolution requirements}

