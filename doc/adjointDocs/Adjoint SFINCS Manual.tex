\documentclass[11pt]{amsart}
\usepackage{geometry}                
\geometry{letterpaper}                   
\usepackage{graphicx}
\usepackage{amssymb}
\usepackage{epstopdf}
\usepackage{amsmath}
\usepackage{amssymb}
\usepackage{graphicx}
\usepackage{placeins}
\usepackage{setspace}
\onehalfspacing
\usepackage{fullpage}
\usepackage{commath}
\usepackage{bm}
\usepackage{breqn}
\usepackage{bm}
\usepackage{mathrsfs}
% My definitions
\def\bhat{\ensuremath{\mathbf{\hat{b}}}}
\def\vpar{\ensuremath{v_{||}}}
\def\vperp{\ensuremath{v_{\perp}}}
\newcommand{\partder}[2]{\dfrac{\partial #1}{\partial #2}} % partial der.
\newcommand{\der}[2]{\dfrac{d #1}{d #2}} % partial der.
\DeclareMathOperator{\sign}{sign}
\DeclareMathOperator{\Res}{Res}
% Definitions for code snipets
\usepackage{listings}
\usepackage{color}
\definecolor{dkgreen}{rgb}{0,0.6,0}
\definecolor{gray}{rgb}{0.5,0.5,0.5}
\definecolor{mauve}{rgb}{0.58,0,0.82}
\lstset{frame=tb,
  language=Matlab,
  aboveskip=3mm,
  belowskip=3mm,
  showstringspaces=false,
  columns=flexible,
  basicstyle={\small\ttfamily},
  numbers=none,
  numberstyle=\tiny\color{gray},
  keywordstyle=\color{blue},
  commentstyle=\color{dkgreen},
  stringstyle=\color{mauve},
  breaklines=true,
  breakatwhitespace=true,
  tabsize=3
}
\begin{document}
\author{Elizabeth Paul}
\date{July 25th, 2017}
\title{Manual for Adjoint SFINCS}
\maketitle

\section{Input Arguments}
As described in the \texttt{SFINCSUserManual}, the sensitivity will be computed when \texttt{sensitivityOption > 1}. There is also a flag to determine the desired fluxes to compute sensitivity of, \texttt{adjointRHSOption}, and for which species the sensitivity of the fluxes will be computed, \texttt{adjointRHSSpeciesOption}. Adjoint SFINCS must also be run with the following input parameters. 
\begin{itemize}
\item \texttt{geometryScheme == 5} (must also be stellarator symmetric)
\item \texttt{includePhi1 == .false.}
\item \texttt{EParallelHat == 0}
\item \texttt{useDKESExBDrift == .false.} 
\item \texttt{magneticDriftScheme == 0}
%\item \texttt{force0RadialCurrentInEquilibrium = .true.}
\item \texttt{includeXDotTerm == .true.} 
\item \texttt{constraintScheme == 1}
\item \texttt{collisionOperator == 0}
\end{itemize}

\section{Computing Geometrical Factors}
We need to compute the derivatives of the components of $\bm{B}$ $\left( \hat{B}, \hat{B}^{\theta}, \hat{B}^{\zeta}, \hat{B}_{\theta}, \hat{B}_{\zeta}, \hat{D}, \partder{\hat{B}}{\theta}, \partder{\hat{B}}{\zeta} \right)$, with respect to their Fourier components ($\hat{B}_{m,n}$, $\hat{B}_{m,n}^{\theta}$, $\hat{B}_{m,n}^{\zeta}$, $\hat{B}^{m,n}_{\theta}$, $\hat{B}^{m,n}_{\zeta}$, $\hat{g}_{m,n}$). The arrays \texttt{dBHatdFourier}, \texttt{dBHat_sub_thetadFourier}, \texttt{dBHat_sub_zetadFourier}, \texttt{dBHat_sup_thetadFourier}, \texttt{dBHat_sup_zetadFourier}, \texttt{dDHatdFourier}, \texttt{dBHatdthetadFourier}, and \texttt{dBHatdzetadFourier} contain these derivatives. These arrays are allocated and populated in \texttt{computeBHat_VMEC}. I also allocate and populate arrays \texttt{ns} and \texttt{ms} which keep track of which modes are kept in the Fourier spectrum. 


\section{Computing Inner Products}
The subroutines \texttt{populatedMatrixdLambda} and \texttt{populatedRHSdLambda} populate $\partder{\mathbb{L}}{\lambda}$ and $\partder{\mathbb{S}}{\lambda}$ for a given Fourier component ($\hat{B}, \hat{B}^{\theta}, \hat{B}^{\zeta}, \hat{B}_{\theta}, \hat{B}_{\zeta}, \hat{D}$) and mode number. These will be called by a subroutine which performs an inner product. 

\section{Constructing Adjoint Matrices}

\texttt{populateMatrix} has been modified so that the adjoint matrix preconditioner and Jacobian can be evaluated using \texttt{whichMatrix = 4} (adjoint Jacobian) or \texttt{whichMatrix = 5}. 

\end{document}  